\thispagestyle{plain}
\section*{Kurzfassung}

\textbf{\large Empfohlen wird die Verwendung dieser Vorlage mit der jeweils aktuellsten TeXLive Version (Linux, Windows) bzw. MacTeX Version (MacOS).
Aktuell ist dies TeXLive 2014. Download hier: 
Wichtig ist auch, dass die Source-Dateien UTF-8 kodiert sind. Dies
ist nur unter Windows ein Problem, benutzen Sie einen Editor, der
utf-8 unterstützt (z.B. TexMaker ab V4, notepad++, sublime).
}

\href{https://www.tug.org/texlive/}{\textbf{\large https://www.tug.org/texlive/}}

Eine aktuelle Version dieser Vorlage gibt es unter 

\href{https://github.com/MaxNoe/VorlageBachelorArbeit}{www.github.com/MaxNoe/VorlageBachelorArbeit}.

Eine Variante, in der bestimmte Elemente in TU-Farben gehalten sind, steht unter 

\href{https://github.com/MaxNoe/VorlageBachelorArbeit/tree/tu-farben}{www.github.com/MaxNoe/VorlageBachelorArbeit/tree/tu-farben}  

zur Verfügung.

Falls es Probleme mit der Vorlage gibt, einfach ein \emph{Issue} auf GitHub aufmachen oder eine Email an
\href{mailto:maximilian.noethe@tu-dortmund.de}{maximilian.noethe@tu-dortmund.de} schreiben.


Hier steht eine Kurzfassung der Arbeit in deutscher Sprache inklusive der Zusammenfassung der
Ergebnisse.
Zusammen mit der englischen Zusammenfassung muss sie auf diese Seite passen.

\section*{Abstract}

The abstract is a short summary of the thesis in English, together with the German summary it has to fit on this page.
