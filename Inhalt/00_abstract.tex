\thispagestyle{plain}

\section*{Hinweise}
Empfohlen wird die Verwendung dieser Vorlage mit der jeweils aktuellsten TeXLive Version (Linux, Windows) bzw. MacTeX Version (MacOS).
Aktuell ist dies TeXLive 2014. Download hier: 
\begin{center}
  \href{https://www.tug.org/texlive/}{https://www.tug.org/texlive/}
\end{center}

Achten Sie auch auf die Kodierung der Quelldateien.
Bei Verwendung von Xe\LaTeX\ oder Lua\LaTeX\ (empfohlen) müssen die
Quelldateien UTF-8 kodiert sein.
Bei Verwendung von pdf\LaTeX\ nutzen Sie die Pakete \texttt{inputenc} und \texttt{fontenc} mit der korrekten Wahl der Kodierungen.

Eine aktuelle Version dieser Vorlage gibt es unter 
\begin{center}
  \href{https://github.com/MaxNoe/TUDoThesis}{www.github.com/MaxNoe/TUDoThesis}
\end{center}
zur Verfügung.

Für Rückmeldungen und bei Problemen mit der Klasse oder Vorlage, bitte ein \emph{Issue} auf GitHub aufmachen oder eine Email an
\href{mailto:maximilian.noethe@tu-dortmund.de}{maximilian.noethe@tu-dortmund.de} schreiben.

Wenn Sie die Dokumentenklasse mit der Option \texttt{tucolor} laden, werden verschiedene Elemente in TU-Grün gesetzt.

\section*{Kurzfassung}
Hier steht eine Kurzfassung der Arbeit in deutscher Sprache inklusive der Zusammenfassung der
Ergebnisse.
Zusammen mit der englischen Zusammenfassung muss sie auf diese Seite passen.

\section*{Abstract}
\begin{english}
The abstract is a short summary of the thesis in English, together with the German summary it has to fit on this page.
\end{english}
